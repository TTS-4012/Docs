\documentclass{article}

\usepackage[a4paper, margin=0.5in,]{geometry}

\usepackage{graphicx}
\usepackage[dvipsnames]{xcolor}
\usepackage{setspace}


\usepackage{xepersian}
\settextfont[Scale=1.3]{Vazirmatn-Medium}
\linespread{1.7}
\pagestyle{empty}

\selectfont

\begin{document}
	\begin{center}
		\Large
		\textbf{به نام خدا}

		~\\
		~\\
		~\\
		~\\
	\includegraphics[width=0.3\textwidth]{assets/IUST_Logo.png}
	~\\
	~\\
	~\\
	
	داک امکان سنجی 
	( Feasibilitty Study )
	~\\
	~\\
	~\\
	
\textcolor{BlueViolet}{\textbf{اعضای گروه:} }\\
	مبین آزادانی \\
	علیرضا دوستی مهر \\
	محمدجواد جلیلوند \\
	مرتضی ملکی نژاد شوشتری \\
	رضا یزدی \\
	دانیال یگانه
	\end{center}
	\newpage
	\tableofcontents
	\newpage
	\section{معرفی پروژه}
	در جامعه برنامه نویسان یکی از مهم ترین راه های یادگیری حل مسائل متنوع و دیدن راه های مختلف برای حل یک سوال است. علاوه بر این بسیاری از شرکت ها برای استخدام با داوطلبان زیادی مواجه هستند که طی کردن فرآیند مصاحبه برای هرنفر فرآیندی طولانی و هزینه بر است. ما تصمیم گرفتیم که یک سیستم داوری سوالات برنامه نویسی طراحی کنیم که هم به برنامه نویس ها برای حل سوال کمک کند و هم این امکان را داشته باشد که بتوان در آن مسابقه طراحی کرد و به این طریق شرکت ها بتوانند قبل از فرآیند مصاحبه و یا چک کردن رزومه معیار سنجش راحت تری برای انتخاب نیروهای جدید داشته باشند.
	از اهداف این پروژه میتوان به این موارد اشاره کرد (فیچرست کامل پروژه در بخش اسکوپ پروژه آمده).
	\begin{itemize}
		\item امکان ساخت راحت مسابقه
		\item امکان تمرین و حل سوالات متعدد 
		\item افزایش یادگیری استفاده کننده برنامه به وسیله آموزش هایی که میتوان در کنار سوالات قرار داد
		\item امکان تست جواب و در صورت خطا دیدن دلیل آن
	\end{itemize}
	
	\section{جامعه هدف}
	جامعه هدف این برنامه چند گروه عمده هستند. اول برنامه نویسانی که به دنبال یافتن و حل مسئله برای رشد توانایی خود هستند. دوم دانشگاه ها، مراکز آموزشی و شرکت هایی که به دلایل مختلفی به پلتفرمی برای برگزاری مسابقه برای سنجش توانایی برنامه نویسان نیاز دارند و سوم برنامه نویسانی که به دنبال کار میگردند و از این راه می توانند با شرکت مورد نظر خود ارتباط برقرار کنند.
	
	از آنجایی که امکان طرح سوال با سختی های مختلفی وجود دارد، این برنامه محدودیت خاصی بر اساس توانایی افراد ندارد و تنها شاید سوالات مربوط به یک سطح بیشتر باشند، همچنین از آنجا که در بخش سوالات (و نه در مسابقه) یک بخش آموزشی قرار دارد امکان حل سوال هایی که کمی سخت تر هستند و با راهنمایی کمی حل می شوند نیز وجود دارد.

	\section{محصولات مشابه در بازار}
	قبل از اینکه به مخصولات مشابه پرداخته شود، نکته ای که باید توجه شود این است که از آنجا که جامعه هدف این نوع برنامه ها عموم نیستند پس تعداد کاربر نیز باید در حیطه خودش مقایسه شود. 
	\begin{enumerate}
		
		\item
		
		 Codeforces
		
		موفقیت ها:
		\begin{itemize}
			\item دارای بیش از ۶۰۰۰۰۰ کاربر
			\item یکی از مهم ترین سایت ها در زمینه competitive programming
			\item بلاگ کدفرسز تبدیل به یک محل گفتگو برای competitive programmer ها شده 
			\item ارتباط با دانشگاه ها و شرکت ها از طریق برگزاری مسابقات و بخش gym سایت
		\end{itemize}
		ویژگی ها:
		\begin{itemize}
			\item برگذاری منظم مسابقات در سطوح مختلف
			\item وجود tutorial در کنار سوالات در بخش problemset
			\item وجود بخش gym برای کلاس های مختلف
			\item وجود امکان hack بعد از مسابقه برای به چالش کشیدن کد سایر شرکت کنندگان
			\item سیستم ریتینگ بر اساس عملکرد شرکت کنندگان در مسابقه
			\item بلاگ فنی که در آن کاربران مختلف میتوانند مطلب بنویسند
		\end{itemize}
		
		\item 
		
		Quera
		
		موفقیت ها
		\begin{itemize}
			\item به عنوان معروف ترین سیستم داوری ایرانی، کوئرا با بسیاری از شرکت ها همکاری داشته 
			\item بسیاری از دانشگاه های سطح بالای کشور از کوئرا برای اهداف آموزشی استفاده می کنند 
			\item کوئرا دوره ها و رویداد های مختلفی برگزار کرده تا هم به کسب درآمد بهتری برسد و هم جامعه خود را غنی تر کند
		\end{itemize}
		
		ویژگی ها
		\begin{itemize}
			\item تمرکز زیاد بر روی بخش B2B
			\item برگذاری دوره های آموزشی محتلف بعضا با همکاری شرکت های مختلف
			\item  علاوه بر سوال های الگوریتمی، کوئرا از سوال های تکنولوژی و ... نیز پشتیبانی و سیستم داوری آن بر روی تعدادی از آنها نیز کار میکند
			\item امکان تعریف جواب هم به صورت ثابت و هم به صورت داینامیک
		\end{itemize}
	\end{enumerate}
	
	\section{اسکوپ پروژه}
	\begin{enumerate}
		\item بخش authentication  و management user
			\begin{itemize}
				\item امکان ثبت نام کاربر جدید و ورود کاربر قدیم
				\item امکان تغییر اطلاعاتی مثل نام، شهر، ایمیل، شماره تلفن، عکس پروفایل و ...
			\end{itemize}
		\item بخش  سوالات
			\begin{itemize}
				\item امکان طرح سوال جدید و حذف سوال قدیمی که توسط کاربر طراحی شده
				\item دیدن صورت سوال و توضیحات دیگر مثل نمونه ورودی، خروجی 
				\item امکان سابمیت کردن کد برای تست درست بودن جواب و داوری درست آن (در حال حاظر ما در نظر گرفتیم تنها از یک زبان پشتیبانی کنیم به دلیل چالش های امنیتی مختلف آن)
				\item تشخیص ارور های مختلف برنامه مثل جواب غلط، خطاهای مختلف، ، مصرف زیاد مموری یا طول کشیدن بیش از حد آن و گزارش به کاربر
				\item مشاهده بخش آموزشی سوال (درصورت وجود)
			\end{itemize}
		\item بخش مسابقه
		\begin{itemize}
			\item تعریف مسابقه و اضافه کردن سوالات مختلف به آن
			\item ست کردن محدودیت زمانی روی مسابقه
			\item امکان شرکت در مسابقه توسط کاربران و اجرای سیستم داوری روی کد ها
			\item مشاهده جدول امتیازات مسابقه
		\end{itemize}
	\end{enumerate}
	
	\section{تکنولوژی مورد استفاده}
		\begin{LTR}
			\begin{tabular}{ l l }
				Backend: & Golang \\
				Storage: & Postgres, Redis,  Minio, \\
				Broker: & Nats \\
				Infrastructure: & docker compose \\
				Frontend: & React
			\end{tabular}
		\end{LTR}
		
		\section{برنامه تجاری}
		\begin{itemize}
			\item بازاریابی
			به نظر ما بهترین برنامه برای بازاریابی استفاده برنامه در دانشگاه های مختلف، و انتشار آن در جاهای مختلف است.
			
			یک مسئله دیگری که وجود دارد تعداد کم سیستم های داوری کد آزاد و متن باز است  به این طریق برنامه ما میتواند در محیط هایی که نیاز به یک سیستم داوری سوال برای خود دارند هم استفاده شود (بر خلاف مثلا کوئرا که مسابقه درون خود سیستم برگزار می‌شود)
			\item روش کسب درآمد
			
			شیوه های مختلفی برای کسب درآمد وجود دارد،، یکی تبلیغات است که ما زیاد روی آن اهمیت ندادیم
			
			برنامه اصلی ما برای کسب درآمد پلن های B2B است که شرکت هایی که میخواهند از پلتفرم ما استفاده کنند با توجه به ابعاد و نوع مسابقه قرارداد ببندند و همینطور میتوان از ایونت ها و ... برای کسب درآمد نیز استفاده کرد.
		\end{itemize}
		
		\section{ریسک ها}
		\subsection{منابع انسانی}
		ریسک های منابع انسانی در اوایل پروژه به نسبت کمتر هستند ولی باز مثال هایی مثل حذف درس، بروز اختلاف، بروز مشکل برای یک از اعضا و .... وجود دارد.
		
		راهی که ما اینجا درنظر گرفتیم این بود که در اسکوپ پروژه کمی فیچر ها را کمتر از حدی که به نظر خود میتوانیم در بهترین شرایط در بهترین حالت برسیم در نظر گرفتیم و یک سری از فیچر ها را به فیچرست آپشنال انتقال دادیم.
		یکی دیگر از راه هایی که ما درنظر گرفتیم این است که از وجود knowledge siloos جلوگیری کنیم و اینطور نباشد که یک بخش از پروژه فقط توسط یک نفر قابل درک باشد. در غیر اینصورت با خروج یک نفر از تیم زمانی باید صرف آنبوردینگ بقیه روی آن تکه کد شود.
		
		ریسک های منابع انسانی بعد از ترم اگر بخواهیم پروژه را ادامه بدهیم به نسبت بیشتر است چون ممکن است افراد حس مسئولیت پذیری کمتری کنند، یکی از دلایلی که ما تصمیم گرفتیم از گواهی GPL برای این برنامه استفاده کنیم همین مورد است که سایر افراد خارج از تیم نیز بتوانند روی آن contribute کنند.
		
		\subsection{ریسک های فنی}
		موارد مهمی که برای این مورد به ذهن ما رسید این موارد است:
		\begin{itemize}
			\item obsolete شدن کتابخانه ها و ابزار های ما
			\item طولانی شدن تعداد کد های داوری و ایجاد یک performance bottleneck
			\item وجود تحریم برای استفاده از ابزارهای متعدد
			\item (این مورد میتواند به مدیریت پروژه برگرد و جزو ریسک ها نباشد) غریبه بودن اعضای تیم با این حیطه
		\end{itemize}
		
		برای این موارد رویکردی که ما داریم این است که از ابزار ها و کتابخانه هایی استفاده کنیم که هم جامعه بزرگی داشته باشند و هم تا حد امکان متن باز باشند تا در صورت متوقف شدن آن ابزار در صورت نیاز خودمان بتوانیم تغییرات مورد نیاز را روی آن ها بدهیم
		
		برای تحریم میتوان از ابزارهای مختلفی استفاده کرد که این موضوع را حل کرد که در بحث این داک نمی‌گنجد درواقع مشکل اصلی این مورد بروز کندی و پیچیدگی در فرآیند develop/deploy است و به اصطلاح پروژه را کامل متوقف نمی‌کند.
		\subsection{ریسک های امنیتی}
		با توجه به اینکه این برنامه نیاز است کد ناشناس را بتواند اجرا کند، بحث ریسک های امنیتی اهمیت ویژه ای دارد
		\begin{itemize}
			\item امکان حمله های DDoS و پایین آمدن سیستم
			\item امکان فرستادن کد های مخرب توسط کاربران به هدف 
		\end{itemize}
		
		برای حمله های DDoS چون فعلا برنامه ما در حالت دولوپ است، احتمال این جور حمله ها پایین تر است ولی در کل برای آینده میتوان از تکنیک های DDoS Mitigation مثلا استفاده از WAF و .... یا بهره گیری از خدمات cloud provider خود استفاده کرد
		
		برای کد های مخرب ما کارهای مختلفی برای ایزوله کردن محیط اجرای کد میتوانیم بکنیم مثلا قطع بودن اینترنت در محیط اجرا، اجرای کد ها در یک container جدا و ....
		
		\subsection{رقبای محصول}
		رقبا میتوانند به دو صورت مستقیم و غیر مستقیم برای ما مشکل درست کنند
		\begin{itemize}
			\item بهره گیری از ایده های ما و پیاده سازی فیچر های ما توسط خود
			\item  از آنجا که سایر برنامه ها از ما بزرگ تر هستند، کاربران میتوانند ترجیح بدهند که از آن ها استفاده کنند
			\item وجود سوال های بیشتر و قوی تر در رقبا
		\end{itemize}
		
		\subsection{بازاریابی}
		\begin{itemize}
			\item عدم استفاده توسط کاربران
			\item این پروژه پیچیدگی های فنی ای دارد که ممکن است مخاطب های ما حوصله توضیحات ما درمورد آن را نداشته باشند و فکر کنند که برنامه ما یک فیچر را ندارد یا باگ دارد 
		\end{itemize}
\end{document}