\documentclass{article}

\usepackage[a4paper, margin=0.5in,]{geometry}

\usepackage{graphicx}
\usepackage[dvipsnames]{xcolor}
\usepackage{setspace}


\usepackage{xepersian}
\settextfont[Scale=1.3]{Vazirmatn-Medium}
\linespread{1.7}
\pagestyle{empty}

\selectfont

\begin{document}
	\begin{center}
		\Large
		\textbf{به نام خدا}

		~\\
		~\\
		~\\
		~\\
	\includegraphics[width=0.3\textwidth]{assets/IUST_Logo.png}
	~\\
	~\\
	~\\
	
	داک امکان سنجی 
	( Feasibilitty Study )
	~\\
	~\\
	~\\
	
\textcolor{BlueViolet}{\textbf{اعضای گروه:} }\\
	مبین آزادانی \\
	علیرضا دوستی مهر \\
	محمدجواد جلیلوند \\
	مرتضی ملکی نژاد شوشتری \\
	رضا یزدی \\
	دانیال یگانه
	\end{center}
	\newpage
	\tableofcontents
	\newpage
	\section{معرفی پروژه}
	در جامعه برنامه نویسان یکی از مهم ترین راه های یادگیری حل مسائل متنوع و دیدن راه های مختلف برای حل یک سوال است. علاوه بر این بسیاری از شرکت ها برای استخدام با داوطلبان زیادی مواجه هستند که طی کردن فرآیند مصاحبه برای هرنفر فرآیندی طولانی و هزینه بر است. ما تصمیم گرفتیم که یک سیستم داوری سوالات برنامه نویسی طراحی کنیم که هم به برنامه نویس ها برای حل سوال کمک کند و هم این امکان را داشته باشد که بتوان در آن مسابقه طراحی کرد و به این طریق شرکت ها بتوانند قبل از فرآیند مصاحبه و یا چک کردن رزومه معیار سنجش راحت تری برای انتخاب نیروهای جدید داشته باشند.
	از اهداف این پروژه میتوان به این موارد اشاره کرد (فیچرست کامل پروژه در بخش اسکوپ پروژه آمده).
	\begin{itemize}
		\item امکان ساخت راحت مسابقه
		\item امکان تمرین و حل سوالات متعدد 
		\item افزایش یادگیری استفاده کننده برنامه به وسیله آموزش هایی که میتوان در کنار سوالات قرار داد
		\item امکان تست جواب و در صورت خطا دیدن دلیل آن
	\end{itemize}
	
	\section{جامعه هدف}
	جامعه هدف این برنامه چند گروه عمده هستند. اول برنامه نویسانی که به دنبال یافتن و حل مسئله برای رشد توانایی خود هستند. دوم دانشگاه ها، مراکز آموزشی و شرکت هایی که به دلایل مختلفی به پلتفرمی برای برگزاری مسابقه برای سنجش توانایی برنامه نویسان نیاز دارند و سوم برنامه نویسانی که به دنبال کار میگردند و از این راه می توانند با شرکت مورد نظر خود ارتباط برقرار کنند.
	
	از آنجایی که امکان طرح سوال با سختی های مختلفی وجود دارد، این برنامه محدودیت خاصی بر اساس توانایی افراد ندارد و تنها شاید سوالات مربوط به یک سطح بیشتر باشند، همچنین از آنجا که در بخش سوالات (و نه در مسابقه) یک بخش آموزشی قرار دارد امکان حل سوال هایی که کمی سخت تر هستند و با راهنمایی کمی حل می شوند نیز وجود دارد.

	\section{محصولات مشابه در بازار}
	قبل از اینکه به مخصولات مشابه پرداخته شود، نکته ای که باید توجه شود این است که از آنجا که جامعه هدف این نوع برنامه ها عموم نیستند پس تعداد کاربر نیز باید در حیطه خودش مقایسه شود. 
	\begin{enumerate}
		
		\item
		
		 Codeforces
		
		موفقیت ها:
		\begin{itemize}
			\item دارای بیش از ۶۰۰۰۰۰ کاربر
			\item یکی از مهم ترین سایت ها در زمینه competitive programming
			\item بلاگ کدفرسز تبدیل به یک محل گفتگو برای competitive programmer ها شده 
			\item ارتباط با دانشگاه ها و شرکت ها از طریق برگزاری مسابقات و بخش gym سایت
		\end{itemize}
		ویژگی ها:
		\begin{itemize}
			\item برگذاری منظم مسابقات در سطوح مختلف
			\item وجود tutorial در کنار سوالات در بخش problemset
			\item وجود بخش gym برای کلاس های مختلف
			\item وجود امکان hack بعد از مسابقه برای به چالش کشیدن کد سایر شرکت کنندگان
			\item سیستم ریتینگ بر اساس عملکرد شرکت کنندگان در مسابقه
			\item بلاگ فنی که در آن کاربران مختلف میتوانند مطلب بنویسند
		\end{itemize}
		
		\item 
		
		Quera
		
		موفقیت ها
		\begin{itemize}
			\item به عنوان معروف ترین سیستم داوری ایرانی، کوئرا با بسیاری از شرکت ها همکاری داشته 
			\item بسیاری از دانشگاه های سطح بالای کشور از کوئرا برای اهداف آموزشی استفاده می کنند 
			\item کوئرا دوره ها و رویداد های مختلفی برگزار کرده تا هم به کسب درآمد بهتری برسد و هم جامعه خود را غنی تر کند
		\end{itemize}
		
		ویژگی ها
		\begin{itemize}
			\item تمرکز زیاد بر روی بخش B2B
			\item برگذاری دوره های آموزشی محتلف بعضا با همکاری شرکت های مختلف
			\item  علاوه بر سوال های الگوریتمی، کوئرا از سوال های تکنولوژی و ... نیز پشتیبانی و سیستم داوری آن بر روی تعدادی از آنها نیز کار میکند
			\item امکان تعریف جواب هم به صورت ثابت و هم به صورت داینامیک
		\end{itemize}
	\end{enumerate}
	
	\section{اسکوپ پروژه}
	\begin{enumerate}
		\item بخش authentication  و management user
			\begin{itemize}
				\item امکان ثبت نام کاربر جدید و ورود کاربر قدیم
				\item امکان تغییر اطلاعاتی مثل نام، شهر، ایمیل، شماره تلفن، عکس پروفایل و ...
			\end{itemize}
		\item بخش  سوالات
			\begin{itemize}
				\item امکان طرح سوال جدید و حذف سوال قدیمی که توسط کاربر طراحی شده
				\item دیدن صورت سوال و توضیحات دیگر مثل نمونه ورودی، خروجی 
				\item امکان سابمیت کردن کد برای تست درست بودن جواب و داوری درست آن (در حال حاظر ما در نظر گرفتیم تنها از یک زبان پشتیبانی کنیم به دلیل چالش های امنیتی مختلف آن)
				\item تشخیص ارور های مختلف برنامه مثل جواب غلط، خطاهای مختلف، ، مصرف زیاد مموری یا طول کشیدن بیش از حد آن و گزارش به کاربر
				\item مشاهده بخش آموزشی سوال (درصورت وجود)
			\end{itemize}
		\item بخش مسابقه
		\begin{itemize}
			\item تعریف مسابقه و اضافه کردن سوالات مختلف به آن
			\item ست کردن محدودیت زمانی روی مسابقه
			\item امکان شرکت در مسابقه توسط کاربران و اجرای سیستم داوری روی کد ها
			\item مشاهده جدول امتیازات مسابقه
		\end{itemize}
	\end{enumerate}
	
	\section{تکنولوژی مورد استفاده}
		\begin{LTR}
			\begin{tabular}{ l l }
				Backend: & Golang \\
				Storage: & Postgres, Redis,  Minio, \\
				Broker: & Nats \\
				Infrastructure: & docker compose \\
				Frontend: & React
			\end{tabular}
		\end{LTR}
\end{document}